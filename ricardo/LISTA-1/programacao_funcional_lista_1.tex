\documentclass[a4paper,10pt,twocolumn]{article}

\usepackage{ucs}
\usepackage[utf8x]{inputenc}
\usepackage[portuguese]{babel}
\usepackage{bookman}
\usepackage[T1]{fontenc}

\usepackage{amsmath}
\usepackage{amsfonts}
\usepackage{amssymb}
%\usepackage[scriptsize]{caption}
%\usepackage{graphicx}
%\usepackage{tikz}
%\usepackage{verbatim}
\usepackage{indentfirst}
\usepackage{anysize}
\usepackage{xcolor}

\author{Universidade Federal do Ceará\\
        Campus de Quixadá\\
        Ciência da Computação\\
        Programação Funcional\\
        Prof.$^o$ Ricardo Reis}
\title{Lista de Exercícios I}
\date{\today}

\marginsize{1cm}{1cm}{1cm}{1cm}

\newcommand{\func}[3]{
\boxed{\textbf{\texttt{#1}}} \\
\textsc{Input:} #2 \\
\textsc{Output:} #3
}

\newcommand{\funcx}[4]{
\func{#1}{#2}{#3} \\
\textsc{Ex(s)}: \\
\texttt{#4}
}

\begin{document}
\maketitle

\noindent \textit{Utilizando Haskell, construir as funções seguintes.}

\begin{enumerate}
\item \func{menorDeDois}
           {Dois números, $x$ e $y$}
           {menor valor entre $x$ e $y$}

\item \func{menorDeTres}
           {Três números, $x$, $y$ e $z$}
           {menor valor entre $x$, $y$ e $z$}

\item \func{fatorial}
           {Um natural $n$}
           {O fatorial de $n$}
           
\item \func{fibonacci}
           {Inteiro positivo $n$}
           {$n$-ésimo termo da sequência de Fibonacci (iniciando em com 0 e 1)}

\item \funcx{elemento}
            {Lista $u$ e um natural $n$}
            {$n$-ésimo termo de $u$}
            {elemento 2 [2,7,3,9] ==> 3}

\item \funcx{pertence}
           {Lista $u$ e valor $x$}
           {Verdadeiro se $x \in u$ e falso do contrário}
           {pertence 1 [3,7,4,2] ==> False }
           
\item \func{total}
           {Lista $u$}
           {total de elementos de $u$. Não use função \texttt{length}.}

\item \func{maior}
           {Lista $u$}
           {A chave de valor máximo em $u$. Não usar função \texttt{max}.}
           
\item \funcx{frequencia}
           {Lista $u$ e valor $x$}
           {Retorna o total de ocorrências de $x$ em $u$.}
           {frequencia 5 [4,5,2,1,5,5,9] ==> 3}

\item \funcx{unico}
           {Lista $u$ e valor $x$}
           {Verdadeiro se $x$ ocorre exatamente uma vez em $u$ e falso do contrário}
           {unico 2 [1,2,3,2] ==> False  \\
            unico 2 [3,1] ==> False \\
            unico 2 [2] ==> True}
           
\item \funcx{maioresQue}
           {Número $x$ e uma lista $u$ de números}
           {Sublista de $u$ cujos números sejam maiores que $x$}
           {maioresQue 10 [4 6 30 3 15 3 10 7] ==> [30, 15]}

\item \funcx{concat}
           {Duas listas $a$ e $b$}
           {Concatenação entre $a$ e $b$}
           {concat [] [] ==> [] \\
           (concat [1,2] [3,4] ==> [1,2,3,4]}

\item \funcx{calda}
           {Uma lista $u$}
           {Calda de $u$ ($u$ sem a primeira chave)}
           {calda [1,2,3,4] ==> [2,3,4]}

\item \funcx{corpo}
           {Uma lista $u$}
           {Corpo de $u$ ($u$ sem a última chave) }
           {corpo [1,2,3,4] ==> [1,2,3]}

\item \funcx{unique}
           {Lista $u$ com possíveis chaves repetidas}
           {Lista com as chaves de $u$ sem repetições}
           {[1,2,5,2,5,7,2,5] ==> [1,2,5,7]}

\item \funcx{menores}
           {Natural $n$ e lista $u$}
           {Lista com os $n$ menores elementos de $u$ na ordem que aparecem em $u$}
           {menores 3 [5,3,1,9,7,2] ==> [3,1,2] \\
            menores 5 [6,1,2,3,4]   ==> [6,1,2,3,4] \\
            menores 4 [3,1,2]       ==> [3,1,2]}

\item \func{alter}
           {Inteiro $n$}
           {Lista $[1,-1,2,-2,3,-3,\cdots,n, -n]$}

\item \funcx{reverso}
           {Lista $u$}
           {Lista das chaves de $u$ na ordem inversa.}
           {reverso [1,2,3,4] ==> [4,3,2,1]}

\item \funcx{divide}
           {Lista $u$ e um natural $n$}
           {Tupla de duas listas, $(A,B)$, onde $A$ é formada pelas $n$ primeiras chaves de $u$ e $B$ pelos elementos restantes}
           {divide [1,2,3,4] 2 ==> ([1,2],[3,4]) \\
            divide [1,2,3,4] 0 ==> ([],[1,2,3,4])}

\item \funcx{intercal}
           {Duas listas $a$ e $b$}
           {Lista com os elementos de $a$ e $b$ intercalados}
           {intercal [1,2,3] [7,8,9] ==> [1,7,2,8,3,9] \\
            intercal [1,2,3] [8,9] ==> [1,8,2,9,3] \\
            intercal [] [1,2,6] ==> [1,2,6]}

\item \funcx{uniao}
           {Duas listas $a$ e $b$ sem repetição de chaves}
           {Lista das chaves de $a$ e $b$ sem repetição}
           {uniao [1,2,3] [2,4,6] ==> [1,2,3,4,6] \\
            uniao [4,5] [1] ==> [4,5,1]}

\item \funcx{intersec}
           {Duas listas $a$ e $b$ sem repetição de chaves}
           {Lista das chaves que $a$ e $b$ possuem em comum}
           {intersec [3,6,5,7] [9,7,5,1,3] ==> [3,5,7]}

\item \funcx{sequencia}
           {Dois números naturais $n$ e $m$}
           {Lista $[m, m+1, m+2, \cdots, m+n-1]$}
           {sequencia 0 2 ==> [] \\ 
            sequencia 3 4 ==> [4,5,6] }

\item \func{inserir}
           {Número $x$ e lista, $u$, de números ordenados ascendentemente}
           {Lista de números ordenados ascendentemente, oriunda da inserção apropriada de $x$ em $u$}
           {inserir 3 [2,7,12] ==> [2,3,7,12]}

\item \func{isSorted}
           {Lista de números, $u$}
           {Verdadeiro se $u$ é ordenada e falso do contrário}

\item \funcx{qsort}
           {Lista $u$ de objetos ordenáveis}
           {Lista ordenada das chaves de $u$ pelo método de ordenação rápida}
           {qsort [7,3,5,7,8,4,4] ==> [3,4,4,5,7,7,8]}

\item \funcx{rotEsq}
           {Um natural $n$ e uma lista ou string $S$}
           {Lista $S$ rotacionada $n$ vezes à esquerda}
           {rotEsq 0 "asdfg" \,\, ==> "asdfg" \\
            rotEsq 1 "asdfg" \,\, ==> "sdfga" \\
            rotEsq 3 "asdfg" \,\, ==> "fgasd" \\
            rotEsq 4 "asdfg" \,\, ==> "gasdf"}

\item \funcx{rotDir}
           {Um natural $n$ e uma lista ou string $S$}
           {Lista $S$ rotacionada $n$ vezes à direita }
           {rotDir 0 "asdfg" \,\, ==> "asdfg" \\
            rotDir 1 "asdfg" \,\, ==> "gasdf" \\
            rotDir 3 "asdfg" \,\, ==> "fgasd" \\
            rotDir 4 "asdfg" \,\, ==> "dfgas"}

\item \funcx{upper}
           {Uma string $S$}
           {Versão em caixa alta da string $S$}
           {upper "abc" \,\,==> "ABC" \\
            upper "a Casa Caiu" \,\,==> "A CASA CAIU" \\
            upper "tenho 45 ABCs" \,\,==> "TENHO 45 ABCS"}

\item \funcx{titulo}
           {String $S$}
           {Versão de $S$ contendo todos os caracteres em caixa baixa exceto aqueles que, por serem iniciais de palavras, devem aparecer em caixa alta}
           {titulo "FuLaNo bElTrAnO silva" \,\, ==> "Fulano Beltrano Silva"}

\item \funcx{selec}
           {Uma lista qualquer $u$ e uma lista de posições $P$}
           {Lista das chaves de $u$ cujas posições estão em $P$}
           {selec "abcdef" [0,3,2,3] ==> "adcd"}
           
\item \funcx{isPalind}
           {Uma string qualquer $S$}
           {Verdadeiro se $S$ é um palíndromo e falso do contrário}
           {isPalind "ana" \,\,==> True \\ 
            isPalind "123aa321" \,\,==> True \\
            isPalind "cachorro" \,\,==> False}

\item \func{primo}
           {Um natural $n$}
           {Verdadeiro se $n$ é primo e falso do contrário}

\item \funcx{sdig}
           {Natural $n$}
           {Soma dos dígitos de $n$}
           {sdig 328464584658 ==> 63}

\item \func{bubblesort}
           {Lista ordenável $u$}
           {Versão ordenada de $u$ pelo método de ordenação em bolhas}

\item \funcx{compac}
           {Lista de números $u$}
           {Lista de listas. Cada lista-componente possui um ou dois elementos. Quando possui dois, representa uma sequência de chaves repetidas de $u$, sendo o primeiro valor o total de repetições e o segundo a chave que se repete. Quando possui um elemento contém uma chave de $u$ que não se repete.}
           {compac [2,2,2,3,4,4,2,9,5,2,4,5,5,5] ==> [[3,2],[3],[2,4],[2],[9],[5],[2],[4],[3,5]]}

\item \funcx{splitints}
           {Lista de inteiros positivos, $u$}
           {Tupla de duas listas, $(A,\,B)$, onde $A$ e $B$ são respectivamente  compostos pelos inteiros ímpares e pares de $u$}
           {splitints [1,2,3,4,5,6,7] => ([1,3,5,7],[2,4,6])}

\item \func{perfeito}
           {Número inteiro positivo, $n$}
           {Verdadeiro se $n$ for um quadrado perfeito e falso do contrário (Um \textit{quadrado perfeito} é um número inteiro cuja raiz quadrada é também um número inteiro). Não utilizar operadores ou funções que retornem números reais.}
           {}

\item \funcx{base}
           {Dois inteiros positivos, $n$ e $b$ ($1<b<37$)}
           {Representação na base $b$ do inteiro $n$}
           {base 17 2 ==> "10001" \\
            base 26 16 ==> "1A"}

\item \func{partes}
           {Lista $u$}
           {Conjunto das partes de $u$ (O conjunto das partes de um conjunto $C$ é o conjunto de todos os subconjuntos distintos e possíveis de $C$)}
           {partes [2,3,2,31] ==> [[],[2],[3],[31],[2,2],[2,3],[2,31], 
           [3,31],[2,2,3],[2,2,31],[2,3,31],[2,2,3,31]]}

\end{enumerate}

\end{document}
